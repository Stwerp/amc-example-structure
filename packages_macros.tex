\geometry{letterpaper}

\usepackage{fp}
\usepackage{amsmath}
\usepackage{amssymb}
\usepackage{wrapfig}
\usepackage{multicol}
\usepackage{textcomp}
\usepackage{tikz}
\usepackage{bitset}
\usepackage{graphicx,epstopdf}
\usepackage{listofitems}


% redefine how the questions look
\def\AMCbeginQuestion#1#2{\par\noindent{\bf #1} #2\hspace*{.5em}}
\def\multiSymbole{$\clubsuit$}

% add more space between questions
%\AMCinterIquest=10ex
\AMCinterBquest=10ex % space between questions

%% CUSTOM COMMANDS
% a blank line
%\newcommand{\blank}{$\rule{1cm}{0.15mm}\,\,$}
% inserts a blank line in questions
\newcommand{\blank}{$\rule{5em}{1pt}\,\,$}
% prints out (and sets) the number of points for a question
\newcounter{totpoints}
\newcommand{\pts}[1]{\addtocounter{totpoints}{#1} \scoring{mz=#1} (#1 points)\,} 
% create AMC macro to display points
%\def\AMCsjtTotalPoints#1{\expandafter{\arabic{totpoints} ... won't work yet
% prints out yes and no answer with tt font
\newcommand{\yc}[1]{\correctchoice{\lstinline{#1}}}
\newcommand{\nc}[1]{\wrongchoice{\lstinline{#1}}}


\newcommand{\studentidarea}{%
\AMCcodeHspace=.35em
\AMCcodeVspace=0em
%\AMCcodeBoxSep=.1em
{\setlength{\parindent}{0pt}%
%\hspace*{\fill}%
%Student ID:\hfill\\
\AMCcode{studentid}{8}%
\hspace*{\fill}%
\begin{minipage}[b]{.5\textwidth}%
\vspace*{.5em} Please enter your student ID to the left and \textbf{PRINT} your first and last name below.
%and write your first and last names below.

\vspace{3ex}

\hfill\namefield{\fbox{    
    \begin{minipage}{.9\linewidth}
      Firstname and lastname:
      
      \vspace*{.5cm}\dotfill
      
      \vspace*{.5cm}\dotfill
      \vspace*{1mm}
    \end{minipage}
  }}\hfill\vspace{5ex}\end{minipage}\hspace*{\fill}

}
}

%\setdefaultgroupmode{withreplacement} % fixed randomness
\setdefaultgroupmode{withoutreplacement} % cyclical randomness

\def\AMCIntervalFormat#1#2{[#1,\,#2)}



%%% CUSTOM SET FOR CODE LISTING
\definecolor{mybg}{rgb}{.5, .5, .5}

% Use this so literate appends and doesn't replace!
\makeatletter
\def\lst@Literatekey#1\@nil@{\let\lst@ifxliterate\lst@if
  \expandafter\def\expandafter\lst@literate\expandafter{\lst@literate#1}}
\makeatother
% setup language stuff
\definecolor{ccsword}{RGB}{108,0,60}
\lstdefinestyle{CCSstyle} {%
  language=C,
  basicstyle=\ttfamily\normalsize,
%  literate={^}{\caret}{1},
  literate={~}{$\sim$}{1},
  literate={|}{$\mid$}{1},
  % keywordstyle=\color{blue}\ttfamily,
  keywordstyle=\color{ccsword}\ttfamily\bfseries,
  stringstyle=\color{red}\ttfamily,
  commentstyle=\color{green!50!black}\ttfamily,
  %morecomment=[l][\color{blue}]{\#},
  backgroundcolor = \color{gray!10},
  framexleftmargin=1em,
}

\lstdefinestyle{mingw} {%
  backgroundcolor=\color{white},
  basicstyle=\ttfamily\footnotesize,
  breakatwhitespace=false,
  frame=single,
  %frame=trBL,
  keepspaces=true,
  language=c,
  numbers=none, %none, left, right
  numbersep=10pt,
  numberstyle=\scriptsize\color{mybg},
  showspaces=false,
  showstringspaces=false,
  stepnumber=2,
}
  
\lstset{style=mingw}



% For clearing DoublePage

\usepackage{ifoddpage}
%\newcommand{\MYcleardoublepage}{\checkoddpage\ifoddpage\clearpage~\clearpage\else\clearpage\fi}
\newcommand{\MYcleardoublepage}{\makeatletter%
        \clearpage%
        \ifodd\thepage\else%
          %\ifAMC@automarks\pagestyle{AMCpageEmpty}\fi%
          \hbox{}\clearpage%
        \fi%
        \makeatother%
}


%%%% TAKEN FROM CRUFT:
% plotting
\usepackage{tikz}
\usetikzlibrary{arrows}
\usetikzlibrary{arrows.meta}
\usepackage{pgfplots}
\usepackage{pgfplotstable}
\pgfplotsset{compat=newest}
% photo-copy safe color
\pgfplotscreateplotcyclelist{prnok}{%
{color={rgb,255:red,128;green,128;blue,128}, mark=none, line width=0.5pt}, 
{color={rgb,255:red,159;green,159;blue,159},mark=none, line width=0.5pt},
{color={rgb,255:red,191;green,191;blue,191},mark=none, line width=0.5pt},
}%
% BW style plot cycles
\pgfplotscreateplotcyclelist{bw}{%
  {black!85, mark=none, line width=1.2pt},
  {black!85, mark=none, line width=1.2pt},
  {black!85, mark=none, line width=1.2pt}, 
{black!62,mark=none, line width=1pt},
{black!40, mark=none, line width=1pt},
{black, mark=none, line width=1pt, dashed}, 
{black!55,mark=none, line width=1pt, dashed},
{black!25, mark=none, line width=1pt, dashed},
}%
% MATLAB color plot cycles
\pgfplotscreateplotcyclelist{matlab}{%
{color={rgb,1:red,0;green,0.4470;blue,0.7410}, line width=1pt, mark=none},
{color={rgb,1:red,0.8500;green,0.3250;blue,0.0980}, line width=1pt, mark=none},
{color={rgb,1:red,0.9290;green,0.6940;blue,0.1250}, line width=1pt, mark=none},
{color={rgb,1:red,0.4940;green,0.1840;blue,0.5560}, line width=1pt, mark=none},
{color={rgb,1:red,0.4660;green,0.6740;blue,0.1880}, line width=1pt, mark=none},
{color={rgb,1:red,0.3010;green,0.7450;blue,0.9330}, line width=1pt, mark=none},
{color={rgb,1:red,0.6350;green,0.0780;blue,0.1840}, linew idth=1pt, mark=none}
}

\newlength{\figwidth}
\setlength{\figwidth}{2.5in}
\newlength{\figheight}
\setlength{\figheight}{1.8in}
% Specify default values for axes
\pgfplotsset{every axis/.style={
cycle list name=bw,
width=\figwidth,
height=\figheight,
%axis lines=middle,
xlabel=$t$,
minor tick num = 1,
%grid=both,
%enlargelimits=0.2,
%grid=none,
legend style={legend columns=1,
  cells={anchor=west},
  font=\footnotesize,
  rounded corners=1pt},
legend pos=south east }
}

%% Binary number formatting
% Example:
%   \def\myarray{{15, 13, 12, 11, 10, 9}}
%   \pgfmathrandominteger{\idx}{0}{5}
%   \pgfmathsetmacro\result{\myarray[\idx]}
\newcommand{\printBinaryBits}[3]{%
  \texttt{\bfseries%
    \foreach \i in {#2,...,#3}{%
      \bitsetGet{#1}{\i}%
    }%
  }%
}
\newcommand{\printBinary}[1]{%
  \pgfmathsetmacro\printBinaryResult{#1}%
  \bitsetSetDec{binaryNumber}{\printBinaryResult}%
  \mbox{%
    \printBinaryBits{binaryNumber}{7}{4} %
    \printBinaryBits{binaryNumber}{3}{0}%
  }%
}
\newcommand{\printBinaryFour}[1]{%
  \pgfmathsetmacro\printBinaryResult{#1}%
  \bitsetSetDec{binaryNumber}{\printBinaryResult}%
  \mbox{%
    % \printBinaryBits{binaryNumber}{7}{4} %
    \printBinaryBits{binaryNumber}{3}{0}%
  }%
}
\newcommand{\printBinaryFourWithCarry}[1]{%
  %\pgfmathparse{int(floor( #1 / 2^(4)))}
  \pgfmathsetmacro\printBinaryResult{#1}%
  \bitsetSetDec{binaryNumber}{\printBinaryResult}%
  \mbox{%
    \printBinaryBits{binaryNumber}{4}{4}\,%
    \printBinaryBits{binaryNumber}{3}{0}
  }%
}
\newcommand{\printBinarySixteen}[1]{%
  \pgfmathsetmacro\printBinaryResult{#1}%
  \bitsetSetDec{binaryNumber}{\printBinaryResult}%
  \mbox{%
    \printBinaryBits{binaryNumber}{15}{12}\,%
    \printBinaryBits{binaryNumber}{11}{8}\,%
    \printBinaryBits{binaryNumber}{7}{4}\,%
    \printBinaryBits{binaryNumber}{3}{0}%
  }%
} 